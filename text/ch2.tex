\hinttext{!!!ACTION REQUIRED!!!}
\hinttext{From this point forward until the conclusion, everything becomes pretty individual. The structure I defined is generic and will most likely have to be adapted. I suggest that you skim through the pages and then clear the files \texttt{text/ch2.tex} to \texttt{text/ch7.tex} before you start writing.}

\noindent The beginning of each section should start with a paragraph that tells the reader how the chapter is organized.

\section{First Background Topic}
\label{s:First-Background-Topic}

Provide background information that will help potential readers to understand your research. It is up to you to decide the volume and content of this chapter.

At the time you start writing your thesis you have probably already published novel works and become an expert in your field of research. You may find it difficulty to assume the perspective of a less experienced reader.

Your potential audience is predominately academic and works on tangentially related things. Thus, you may assume that a typical reader has successfully completed all compulsory undergraduate-degree subjects for a related degree. For example, if you do a PhD in computer science or a related discipline, you may assume that the reader knows what the difference between the declarative and imperative programming paradigm is, or what the difference between a linked-list and a tree is.

\section{Second Background Topic}
\label{s:Second-Background-Topic}

Are there people in your department that teach an elective or graduate-level subject that relates to your field of research? If yes, check how they structure their lecture! Tap into the knowledge of the more experienced people around you. Discuss with your supervisors or research coordinator what background information they regard as relevant. Sometimes, they may have a very clear idea.


\section{Acronym Example}\label{sec:acro_example}

This examples comes from \href{https://mirror.aarnet.edu.au/pub/CTAN/macros/latex/contrib/acronym/acronym.pdf}{here}

In the early nineties, \acs{GSM} was deployed in many European countries. \ac{GSM} offered for the first time international roaming for mobile subscribers. The \acs{GSM}’s use of \ac{TDMA} as its communication standard was debated at length. And every now and then there are big discussion whether \ac{CDMA} should have been chosen over \ac{TDMA}.

\section{Summary}
\label{s:Background-Summary}

The final section of each chapter should summarize the chapter. In comparison to the chapter, the summary should be short ($\frac{1}{2}$ to $2$ pages is normal).